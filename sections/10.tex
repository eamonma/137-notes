\documentclass[../13.tex]{subfiles}
\begin{document}
Let \(f\) be a continuous function on \([1, \infty)\). What is the relation between $\sum_{n = 1}^{\infty} f(n)$ and $\int_{1}^{\infty} {f(x)} dx$ ? \\
A series and an improper integral are defined in a similar manner.
\begin{align*}
    \sum_{n = 1}^{\infty} f(n) = \lim_{k\to\infty} \sum_{n = 1}^{k} f(n) \qquad \qquad \int_{1}^{\infty} {f(x)} dx = \lim_{b\to\infty} \int_{1}^{b} {f(x)} dx
\end{align*}
Is there a relation? There is a tool for integrals that we don't have for series in that we can calculate the value of an improper integral---if we have an expression for its antiderivative. \\
I will assume: \begin{itemize}
    \item \(f\) is positive -- then \DS{\sum_{n = 1}^{\infty} f(n)} and \DS{\int_{1}^{\infty} {f(x)} dx } are convergent or \(\infty\) \begin{quote}
              There is not a possibility of oscillation
          \end{quote}
    \item \(f\) is decreasing
\end{itemize}
Plan: find a relation between proper integrals and finite sums, then take the limit. \\
From a graph, we can tell that \(\int_{1}^{5} {f(x)} dx\) is bounded below by \(\sum_{n = 1}^{4} f(n)\). In order to be able to do this, we need the function to be decreasing, so that the \(f(x)\) at the left endpoint would be the maximum on that interval. \\
We can also do the same to bound the area below. Then \(\int_{1}^{5} {f(x)} dx \geq \sum_{n = 2}^{5} f(n)\). Then, we arrive at the expression \begin{align*}
    \sum_{n = 2}^{5} f(n) \leq      & \int_{1}^{5} {f(x)} dx \leq \sum_{n = 1}^{4} f(n)                                                                                                        \\
    \sum_{n = 2}^{N} f(n) \leq      & \int_{1}^{N} {f(x)} dx \leq \sum_{n = 1}^{N-1} f(n) \quad \text{since we could've picked any other integer}                                              \\
    \sum_{n = 2}^{\infty} f(n) \leq & \int_{1}^{\infty} {f(x)} dx \leq \sum_{n = 1}^{\infty} f(n)                                            \quad \text{by taking the limit as } n \to \infty \\
\end{align*}
Then, if the right series---which bounds the integral above---is convergent, then the integral must also be convergent. If the integral---which bounds the left series above---is convergent, then that series must also be convergent. Since the function is positive, this has to be convergent or \(\infty\). Then, we have proven that \thm{
Let \(a \in \R\)\\
Let \(f\) be a \emph{continuous, positive, decreasing} function on [a, \infty) \\
Then \begin{align*}
    \int_{a}^{\infty} {f(x)} dx \quad \text{is convergent} \quad \Longleftrightarrow \quad \sum_{n}^{\infty} f(n) \quad \text{is convergent}
\end{align*}
Denoted with \(\int_{a}^{\infty} {f(x)} dx \sim \sum_{n}^{\infty} f(n)\)
}\newpage
\end{document}
