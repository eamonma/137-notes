\documentclass[../13.tex]{subfiles}
\begin{document}
We cannot take infinite sums as if they were finite sums. \\
e.g.: \begin{align*}
  S      & = \sum_{n=0}^{\infty} x^n = 1 + x + x^2 + x^3 + \dots \\
  xS     & = x + x^2 + x^3 + x^4+n\dots                          \\
  S - xS & = 1                                                   \\
  S      & = \frac{1}{1-x}                                       \\
  \intertext{When \DS{ x = 2: }}
  S      & = \frac{1}{1-2} = -1                                  \\
  S      & = 1 + 2 + 4 + 8 + \dots
\end{align*}
e.g.2: \begin{align*}
  T & = \sum_{n = 0}^{\infty} (-1)^n     \\
    & = 1 -1 + 1-1 + 1 -1 + 1 -1 + \dots
\end{align*}
``We can group the terms like this:''\begin{align*}
  T & = (1-1) + (1-1) + (1-1) + \dots \\
    & =0 + 0 + 0 + \dots = 0
\end{align*}
Alternatively,
\begin{align*}
  T & = 1 (-1 + 1) + (-1 + 1) + (-1 + 1) + \dots \\
    & = 1 + 0 + 0 + 0 + \dots = 1                \\
  T = 0 = 1 ???
\end{align*}

\subsection*{Infinite sums: the right way}
\begin{itemize}
  \item What does adding up infinitely many numbers mean? Define what an infinite sum---a ``series''---is.
  \item When is a series equal to a number? i.e., when is a series convergent?
  \item Which properties of finite sums carry over to infinite sums?
\end{itemize}
\newpage
\end{document}