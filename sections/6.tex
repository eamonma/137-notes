\documentclass[../13.tex]{subfiles}
\begin{document}
This refers to these two properties:\\
\begin{minipage}{0.5\linewidth}
    \begin{equation*}
        {\sum_{n = 0}^{\infty} (a_n + b_n) = \sum_{n = 0}^{\infty} a_n + \sum_{n = 0}^{\infty} b_n}
    \end{equation*}
\end{minipage}
\begin{minipage}{0.5\linewidth}
    \begin{equation*}
        {\sum_{n = 0}^{\infty} (ca_n)= c \sum_{n = 0}^{\infty} a_n}
    \end{equation*}
\end{minipage}
Finite sums can be reordered at-will, but infinite sums cannot.

\thm{
    \textsc{if} \(\sum_{n = 0}^{\infty} a_n\text{ and }\sum_{n = 0}^{\infty} b_n\) are both convergent, \\
    \textsc{then} \(\sum_{n = 0}^{\infty} (a_n + b_n)\) is also convergent and
    \begin{equation*}
        \sum_{n = 0}^{\infty} (a_n + b_n) = \sum_{n = 0}^{\infty} a_n + \sum_{n = 0}^{\infty} b_n
    \end{equation*}
}
\thm{Let \DS{c \in \R}. \\
    \textsc{if} \(\sum_{n = 0}^{\infty} a_n\) is convergent, \\
    \textsc{then} \(\sum_{n = 0}^{\infty} (ca_n)\) is also convergent, and
    \begin{equation*}
        \sum_{n = 0}^{\infty} (ca_n) = c \sum_{n = 0}^{\infty} a_n
    \end{equation*}}

\begin{proof} for \emph{Theorem 6.1}
    \begin{align*}
        \sum_{n = 0}^{\infty} a_n = \lim_{k\to\infty} S_k, \qquad         & \text{where } S_k = \sum_{n = 0}^{\infty} a_n         \\
        \sum_{n = 0}^{\infty} b_n = \lim_{k\to\infty} T_k, \qquad         & \text{where } T_k = \sum_{n = 0}^{\infty} b_n         \\
        \sum_{n = 0}^{\infty} (a_n + b_n) = \lim_{k\to\infty} R_k, \qquad & \text{where } R_k = \sum_{n = 0}^{\infty} (a_n + b_n) \\
        \text{By properties of finite sums, }                             & \quad R_k = S_k + T_k                                 \\
        \text{By hypothesis, \emph{the first two limits exist}. Then, by the limit laws, }                                        \\
        \lim_{k\to\infty} R_k                                             & \lim_{k\to\infty} S_k + \lim_{k\to\infty} T_k
    \end{align*}
\end{proof}

This is a template for series---how they're proven. \begin{enumerate}
    \item Write the infinite sum as a limit of partial sums---finite sums
    \item Use the properties known to be true for finite sums
    \item Pass to the limit
\end{enumerate}

\newpage
\end{document}