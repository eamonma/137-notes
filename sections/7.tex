\documentclass[../13.tex]{subfiles}
\begin{document}
We often talk about the tail of a series when studying infinite sums. Let's compare \DS{\sum_{n = 0}^{\infty} a_n} and \DS{\sum_{n = 1}^{\infty} a_n}.

\thm{
    \(\sum_{n = 0}^{\infty} a_n\) is convergent \DS{\Longleftrightarrow} \(\sum_{n = 1}^{\infty} a_n\) is convergent \\
    Moreover, in that case
    \begin{align*}
        \sbr{\sum_{n = 0}^{\infty} a_n} = a_0 + \sbr{\sum_{n = 1}^{\infty} a_n}
    \end{align*}
}

Proof involves writing the series as a limit of finite sums, then we know the property to be true for finite sums. \\
Because this is true, we can write \fbox{\DS{\sum_{n}^{\infty} a_n} is convergent / divergent}; in other words, we don't need to specify where it starts. \\
~\\~\\
\fbox{
    \parbox{\textwidth}{\begin{theorem*} ~\\
            Typical theorem: ~\\
            \textsc{if} for all \DS{n \in \N}, \fbox{something about \DS{a_n}} \\
            \textsc{then} the series \(\sum_{n}^{\infty} a_n\) is convergent.
        \end{theorem*}}
}
~\\~\\
\fbox{
    \parbox{\textwidth}{\begin{theorem*} ~\\
            Generalized theorem: ~\\
            \textsc{if} \DS{\exists n_0 \in \N \textsc{ s.t. } \forall n \in \N, \quad n \geq n_0 \implies} \fbox{something about \DS{a_n}} \\
            \textsc{then} the series \(\sum_{n}^{\infty} a_n\) is convergent.
        \end{theorem*}}
}


\newpage
\end{document}