\input{~/preamble.tex}

\usepackage{blindtext}


\usepackage{subfiles}

\sectionfont{\color{blue}\selectfont}
\subsectionfont{\color{green}\selectfont}

\newcommand{\eg}{\textbf{e.g.}~}
\newcommand{\abs}[1]{\left\lvert #1 \right\rvert}

\begin{document}
\newtheorem{theorem}{Theorem}[section]
\newtheorem*{theorem*}{Theorem}
\newtheorem{definition}{Definition}[section]
{\LARGE \textbf{Unit 13: Series}}
\thispagestyle{empty}
\tableofcontents
\newpage
\clearpage
\setcounter{page}{1}
\section{Power series: an example}
\eg I want to define a function with this equation: \[g(x) = \sum_{n = 1}^{\infty} \frac{x^n}{n3^n}\]
For which \(x \in \R\) is \(g(x)\) convergent? We can use the Ratio Test.
\begin{align*}
    \text{Call } a_n                                       & = \frac{x^n}{n3^n}                                                      \\
    L = \lim_{n\to\infty} \cfrac{\abs{a_{n+1}}}{\abs{a_n}} & = \cfrac{\abs{\cfrac{x^{n+1}}{(n+1) 3^{n+1}}}}{\abs{\cfrac{x^n}{n3^n}}} \\
                                                           & = \frac{\abs{x}}{3}
\end{align*}
\begin{itemize}
    \item If \(\abs{x} < 3\), then \(L = \frac{\abs{x}}{3} < 1\). By Ratio Test, \(g(x)\) is absolutely convergent.
    \item If \(\abs{x} > 3\), then \(L = \frac{\abs{x}}{3} > 1\). By Ratio test, \(g(x)\) is divergent.
\end{itemize}
We don't know what happens at \(x = -3\) or \(x = 3\) yet. \\
\begin{align*}
    g(3)  & = \sum_{n = 1}^{\infty} \frac{3^n}{n3^n} = \frac{1}{n}  = \infty \text{ (p-series with \DS{p = 1})} \\
    g(-3) & = \sum_{n = 1}^{\infty} \frac{(-3)^n}{n3^n} = \frac{(-1)^n}{n} \text{ convergent (by \textsc{ast})} \\
\end{align*}
Then, at \(x = -3\), \(g(x)\) is conditionally convergent, and \(3\), it is divergent. \\
To answer the original question, the domain of \(g = [-3, 3)\) = the \textsc{interval of convergence}. \(3\) = the \textsc{radius of convergence}.
\newpage



\section{Power series: the main theorem}
\subsection*{Motivation}
\begin{itemize}
    \item Polynomials are nice
    \item What about ``infinite polynomials''? \\
          \[f(x) = c_0 + c_1x + c_2x^2 + c_3x^3 + \dots\]
          \[\text{or} \quad f(x) = c_0 + c_1 (x - a) + c_2 (x - a)^2 + c_3 (x - a)^3 + \dots\]
    \item \eg: \begin{itemize}
              \item \(g(x) = \sum_{n = 1}^{\infty} \frac{x^n}{n3^n}\) has domain \([-3, 3)\)
              \item \(h(x) = \sum_{n = 0}^{\infty} x^n\) has domain \((-1, 1)\)
          \end{itemize}
\end{itemize}
\df{
    Let \(a \in \R\). \\
    A \underline{power series centered at \(a\)} is a function \(f\) defined by an equation like \begin{align*}
        f(x) = \sum_{n = 0}^{\infty} c_n(x-a)^n = c_0 + c_1(x - a) + c_2(x - a)^2 + \dots
    \end{align*}
    where \(c_0, c_1, c_2, \dots \in \R\).
}

\begin{center}
    Domain \(f = \{x \in \R ~ : ~ \text{teh series \(f(x)\) is convergent}\}\)
\end{center}
Note: \(a \in\) Domain \(f\) \\
Ultimate goal: write common functions as power series.
\thm{
    Let \(f(x) = \sum_{n = 0}^{\infty} c_n (x - a)^n\) be a power series centered at \(a \in \R\). \begin{enumerate}
        \item The domain of \(f\) is an interval centered at a: \begin{align*}
                   & (a - R, a + R) \quad (a - R, a + R] \quad \R    \\
                   & [a - R, a + R] \quad [a - R, a + R) \quad \{a\}
              \end{align*} \begin{itemize}
                  \item We call this domain the \underline{interval of convergence} (\textsc{ic}) of \(f\).
                  \item We call its radius the \underline{radius of convergence}. \(\quad 0 \leq R \leq \infty\)
              \end{itemize}
        \item \begin{itemize}
                  \item In the \textbf{interior} of the \textsc{ic}, the series is absolutely convergent.
                  \item In the \textbf{exterior} of the \textsc{ic}, the series is divergent.
                  \item At the endpoints (if any), anything may happen.
              \end{itemize}
        \item In the interior of the \textsc{ic}, power series can be ``treated like polynomials''. They can be added, multiplied, composed\dots \\
              In particular, they can be differentiated or integrated ``term by term'', and this does not change the radius of convergence.
    \end{enumerate}
}

\begin{alignat*}{2}
    f(x)                   & = \sum_{n = 0}^{\infty} c_n x^n             &  & = c_0 + c_1x + c_2x^2 + c_3x^3 + \dots                 \\
    f'(x)                  & = \sum_{n = 0}^{\infty} c_n n x^{n-1}       &  & = c_1 + 2c_2x + 3c_3x^2 + \dots                        \\
    \int_{0}^{x} {f(t)} dt & = \sum_{n = 0}^{\infty} \frac{x^{n+1}}{n+1} &  & = c_0x + c_1 \frac{x^2}{2} + c_2 \frac{x^3}{3} + \dots
\end{alignat*}

\textbf{Goals} \begin{enumerate}
    \item Write as many functions as possible as power series \\
          \(\longrightarrow\) Taylor series
    \item Use that to make limits, integrals, estimations, differential equations, physics,\dots easier.
\end{enumerate}

\newpage
\section{Taylor polynomials---the definition with the limit}
Goal: approximate functions with polynomials. \\
\(f\): function \\
\(a \in\) domain \(f\) \\
\(P\): polynomial \\

I want \(P(x) \approx f(x)\) when \(x\) is close to \(a\). Example: the tangent line. But what is a ``good approximation near a''? \\
\(R\): "remainder" or "error" \(\quad R(x) = f(x) - P(x)\). I want \(R\) to be small.

\newpage
\end{document}