\documentclass[fontsize=12pt]{article}
\usepackage[utf8]{inputenc}
\usepackage[margin=0.75in]{geometry}
\usepackage{fancyvrb, amsmath, amsthm, amssymb, setspace, amssymb, graphicx, enumitem, amsfonts, amssymb, ifthen, geometry}
\usepackage[style=iso]{datetime2}
\usepackage[T1]{fontenc}
\newcommand {\DS} [1] {${\displaystyle #1}$}
\newcommand{\R}{\mathbb{R}} \newcommand{\C}{\mathbb{C}} \newcommand{\Z}{\mathbb{Z}} \newcommand{\N}{\mathbb{N}} \renewcommand{\qedsymbol}{$\blacksquare$}

% auto brackets
\usepackage{mathtools}
\DeclarePairedDelimiter\autobracket{(}{)} \newcommand{\br}[1]{\autobracket*{#1}}
\DeclarePairedDelimiter\autosquarebracket{[}{]} \newcommand{\sbr}[1]{\autosquarebracket*{#1}}


\title{}
\author{}
\date{\today}

\usepackage{fontspec}
\setmainfont{Libertinus Serif}
\setsansfont{Libertinus Sans}

\setstretch{1.5}
\begin{document}
\subsection*{Infinite sums: a cautionary tale}
We cannot take infinite sums as if they were finite sums. \\
e.g.: \begin{align*}
    S      & = \sum_{n=0}^{\infty} x^n = 1 + x + x^2 + x^3 + \dots \\
    xS     & = x + x^2 + x^3 + x^4+n\dots                          \\
    S - xS & = 1                                                   \\
    S      & = \frac{1}{1-x}                                       \\
    \intertext{When \DS{ x = 2: }}
    S      & = \frac{1}{1-2} = -1                                  \\
    S      & = 1 + 2 + 4 + 8 + \dots
\end{align*}
e.g.2: \begin{align*}
    T & = \sum_{n = 0}^{\infty} (-1)^n     \\
      & = 1 -1 + 1-1 + 1 -1 + 1 -1 + \dots
\end{align*}
\end{document}
